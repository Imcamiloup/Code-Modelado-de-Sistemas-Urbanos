\documentclass[final]{beamer}
\usepackage[size=custom,width=80,height=120,scale=1.0]{beamerposter}
\usetheme{confposter}
\usecolortheme{seahorse}
\usepackage[utf8]{inputenc}
\usepackage[T1]{fontenc}
\usepackage[spanish]{babel}
\usepackage{amsmath, amsfonts, amssymb}
\usepackage{graphicx}
\usepackage{booktabs}
\usepackage{hyperref}

\graphicspath{{img/}}

\title{Modelo Integrado de Sistemas Urbanos Acoplados\\
Caso de Estudio: Bogotá D.C.}
\author{Luis Camilo Gómez Rodríguez}
\institute{Universidad Nacional de Colombia | Ingeniería de Sistemas | 2025}

\begin{document}
\begin{frame}[t]

% EL HEADLINE SE GENERA AUTOMÁTICAMENTE POR EL .sty

\begin{columns}[t]

% -------- COLUMNA IZQUIERDA --------
\begin{column}{0.47\paperwidth}

\begin{block}{Contexto y Objetivo}
\large
Sistemas urbanos con retroalimentaciones entre \textbf{Población (P)}, \textbf{Desigualdad (F)}, \textbf{Bienestar (W)}, \textbf{Huella Urbana (U)} e \textbf{Infraestructura Ecológica (E)}. 

\vskip0.5cm
\textbf{Objetivo:} Calibrar y validar un sistema acoplado P–F–W para explicar y proyectar la dinámica urbana de Bogotá.
\end{block}

\begin{block}{Sistema de Ecuaciones Diferenciales}
\large
Versión calibrada con U, E exógenas suavizadas (polinomios grado 2 y 3):
\[
\begin{aligned}
\dot P &= rP\left(1-\frac{P}{K}\right)+\gamma_F F \\[0.4cm]
\dot F &= -\alpha_F F+\beta_F W \\[0.4cm]
\dot W &= \alpha_W P+\beta_W(U+E)+\gamma_W W-\delta_W F + c_0
\end{aligned}
\]
\end{block}

\begin{block}{Metodología}
\large
\begin{itemize}
\setlength\itemsep{0.4cm}
\item \textbf{Datos:} Series anuales 2012–2024 (DANE, SDP Bogotá)
\item \textbf{Preproceso:} Normalización z-score, suavizado polinomial
\item \textbf{Optimización:} Híbrida Differential Evolution + L-BFGS-B
\item \textbf{Validación:} RMSE, MAE, R² por variable
\end{itemize}
\end{block}

\begin{block}{Parámetros Estimados (escala normalizada)}
\large
\centering
\begin{tabular}{l c l}
\toprule
\textbf{Parámetro} & \textbf{Valor} & \textbf{Interpretación} \\
\midrule
$r$ & 0.500 & Tasa máx. crecimiento P \\
$K$ & 1.557 & Capacidad de carga \\
$\gamma_F$ & $-1.686$ & F reduce P \\
$\alpha_F$ & 0.225 & Relajación de F \\
$\beta_F$ & 0.010 & W → F (débil) \\
$\alpha_W$ & $-0.010$ & P → W (negativo) \\
$\beta_W$ & $-0.080$ & U+E en W \\
$\gamma_W$ & $-1.000$ & Amortiguamiento \\
$\delta_W$ & 1.000 & F penaliza W \\
\bottomrule
\end{tabular}
\end{block}

\begin{block}{Validación: Ajuste Histórico}
\large
\textbf{Coeficientes R²:} P = 0.96 (excelente) | F ≈ 0.17 | W ≈ 0.27
\vskip0.5cm
\centering
\includegraphics[width=0.95\linewidth]{validacion_sistema_logistico.png}
\vskip0.3cm
\raggedright
Modelo captura bien la tendencia poblacional; F y W muestran alta volatilidad.
\end{block}

\end{column}

% -------- COLUMNA DERECHA --------
\begin{column}{0.47\paperwidth}

\begin{block}{Análisis de Estabilidad}
\large
\textbf{Jacobiano evaluado en puntos fijos:}
\[
J = \begin{bmatrix}
r(1-2P/K) & \gamma_F & 0 \\
0 & -\alpha_F & \beta_F \\
\alpha_W & -\delta_W & \gamma_W
\end{bmatrix}
\]

\vskip0.5cm
\textbf{Resultados:}
\begin{itemize}
\item \textbf{PF1} ($-0.01, -0.003, -0.07$): Silla (inestable)
\item \textcolor{red}{\textbf{PF2} ($1.57, -0.004, -0.09$): Atractor estable}
\end{itemize}

\vskip0.3cm
Eigenvalores PF2: $\lambda_{1,2,3} < 0$ → convergencia asintótica garantizada
\end{block}

\begin{block}{Diagramas de Fase (2D)}
\centering
\begin{minipage}{0.48\linewidth}
\centering
\includegraphics[width=\linewidth]{fase_PW.png}\\[0.2cm]
{\large \textbf{P–W} (F=0)}
\end{minipage}\hfill
\begin{minipage}{0.48\linewidth}
\centering
\includegraphics[width=\linewidth]{fase_PF.png}\\[0.2cm]
{\large \textbf{P–F} (W=0)}
\end{minipage}

\vskip0.8cm
\includegraphics[width=0.6\linewidth]{fase_FW.png}\\[0.2cm]
{\large \textbf{F–W} (P=K/2)}

\vskip0.5cm
\raggedright
\large
Flujos convergen hacia PF2 (rojo). Sistema globalmente estable sin ciclos límite.
\end{block}

\begin{block}{Proyección a 5 Años}
\large
\textbf{Escenario:} U, E constantes; condición inicial = último año (2024)
\vskip0.5cm
\centering
\includegraphics[width=0.95\linewidth]{proyeccion_5anios.png}
\vskip0.3cm
\raggedright
Población converge a saturación (~8M hab.); desigualdad se relaja; bienestar estabiliza bajo.
\end{block}

\begin{block}{Conclusiones}
\large
\begin{itemize}
\setlength\itemsep{0.4cm}
\item Sistema presenta \textbf{atractor único estable} → dinámica predecible
\item Población tiende a K → necesidad de expandir capacidad de carga
\item F penaliza fuertemente W ($\delta_W=1$) → políticas redistributivas urgentes
\item \textbf{Palancas:} aumentar K (infraestructura), fortalecer $\beta_F$ (W→F), reducir $\delta_W$ (F→W)
\end{itemize}
\end{block}

\end{column}
\end{columns}

\end{frame}
\end{document}