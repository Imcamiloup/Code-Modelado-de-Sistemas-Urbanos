% Diapositiva: Cierre y Conclusiones Finales
\begin{frame}{Conclusiones Finales y Perspectivas}
\small

\textbf{Logros principales del estudio:}
\begin{itemize}
  \item ✅ \textbf{Sistema dinámico acoplado calibrado}: Modelo logístico P-F-W con 10 parámetros estimados mediante optimización híbrida (global + local)
  \item ✅ \textbf{Estabilidad demostrada}: Atractor único (PF2) con eigenvalores negativos → convergencia asintótica garantizada
  \item ✅ \textbf{Capacidad predictiva}: $R^2_P = 0.96$ en población, identificación de punto de saturación ($K=1.56$ z-score)
  \item ✅ \textbf{Diagnóstico urbano}: Desigualdad se autorreduce ($\alpha_F>0$), pero bienestar penalizado por $F$ ($\delta_W=1.0$)
\end{itemize}

\vspace{6pt}
\textbf{Limitaciones identificadas:}
\begin{itemize}
  \item ⚠️ Ajuste débil en F ($R^2=0.17$) y W ($R^2=0.27$) → alta volatilidad no capturada
  \item ⚠️ U, E tratadas como exógenas → no captura retroalimentación dinámica completa
  \item ⚠️ $\beta_F$ muy pequeño (0.01) → bienestar no mejora efectivamente desigualdad
  \item ⚠️ $\beta_W<0$ contradice intuición → posible multicolinealidad en datos
\end{itemize}



\end{frame}