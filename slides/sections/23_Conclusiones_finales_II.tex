% Diapositiva: Cierre y Conclusiones Finales
\begin{frame}{Conclusiones Finales y Perspectivas}
\small


\vspace{6pt}
\textbf{Recomendaciones de política basadas en evidencia:}
\begin{enumerate}
  \item \textcolor{blue}{\textbf{Expandir capacidad de carga (K):}} Invertir en infraestructura para evitar saturación poblacional (K actual en límite)
  \item \textcolor{blue}{\textbf{Fortalecer vínculo bienestar-desigualdad:}} Políticas redistributivas para aumentar $\beta_F$ (actualmente 1\%)
  \item \textcolor{blue}{\textbf{Mitigar impacto de F en W:}} Redes de protección social para reducir $\delta_W$ (actualmente penalización 1:1)
\end{enumerate}

\vspace{6pt}
\textbf{Trabajo futuro:}
\begin{itemize}
  \item Extensión a sistema de 5 EDOs ($\dot U, \dot E$ endógenas) con términos no lineales ($F^2$, $PF$)
  \item Análisis de bifurcaciones sobre parámetros críticos ($K$, $\gamma_F$) para escenarios de política
  \item Bootstrap para intervalos de confianza de $\theta$ y eigenvalores (cuantificar incertidumbre)
  \item Validación con datos de otras ciudades (generalización del modelo)
\end{itemize}

\vspace{4pt}
\begin{center}
\textbf{\large Modelo parsimonioso, estable y calibrado con datos reales → herramienta potencial para planeación urbana sostenible}
\end{center}

\end{frame}