% Diapositiva 5: Implementación Computacional
\begin{frame}{Implementación: \texttt{estimacion\_parametros\_sistema.py}}
\small
\textbf{Flujo de ejecución} (describir pipeline modular):
\begin{enumerate}
  \item \texttt{cargar\_normalizado()}: Lee Excel, normaliza, retorna dict (datos listos para uso)
  \item \texttt{suavizar(x, deg)}: Ajuste polinómico de U, E (filtrado previo)
  \item \texttt{edo(t, y, pars, ...)}: Define sistema $\dot P,\dot F,\dot W$ (núcleo dinámico)
  \item \texttt{perdida(pars, ...)}: Integra EDOs, calcula MSE ponderado (función objetivo)
  \item \texttt{estimar()}: Ejecuta optimización, guarda JSON (orquestador principal)
\end{enumerate}

\textbf{Salida:} \texttt{parametros\_sistema\_logistico.json} (10 parámetros + estadísticos normalización)\\[6pt]

\textbf{Tiempo de cómputo:} $\sim$3-5 min (mencionar trade-off precisión/velocidad)\\
\textbf{Librerías clave:} \texttt{scipy.integrate.solve\_ivp}, \texttt{scipy.optimize.differential\_evolution}
\end{frame}
