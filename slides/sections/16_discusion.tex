% Diapositiva 5: Conclusiones I - Dinámica del Sistema
\begin{frame}{Conclusiones I: Dinámica del Sistema}
\small
\textbf{1. Estabilidad global garantizada}
\begin{itemize}
  \item Atractor único PF2 con eigenvalores $\lambda_i < 0$ $\Rightarrow$ \textbf{sistema urbano autorregulado}
  \item No hay ciclos límite (ausencia de $\lambda$ complejos) $\Rightarrow$ convergencia sin oscilaciones indefinidas
  \item \textcolor{darkgreen}{\textbf{Implicación:}} Políticas urbanas eventualmente llevan a equilibrio estable (predictibilidad a largo plazo)
\end{itemize}

\textbf{2. Población tiende a saturación}
\begin{itemize}
  \item $P_{\text{eq}} = 1.57 \approx K = 1.56$ (en z-score) $\Rightarrow$ capacidad de carga alcanzada
  \item \textcolor{orange}{\textbf{Alerta:}} Crecimiento futuro limitado por infraestructura ($K$ fijo)
  \item \textcolor{blue}{\textbf{Recomendación:}} Invertir en $K$ (servicios, vivienda) para permitir expansión sostenible
\end{itemize}

\textbf{3. Desigualdad se autorreduce pero persiste}
\begin{itemize}
  \item $\alpha_F = 0.225 > 0$ $\Rightarrow$ relajación estructural hacia $F \approx 0$ (media histórica)
  \item $\beta_F = 0.01$ (muy pequeño) $\Rightarrow$ \textbf{bienestar no retroalimenta desigualdad efectivamente}
  \item \textcolor{red}{\textbf{Limitación:}} Mejoras en $W$ no reducen significativamente $F$ (desacople parcial)
\end{itemize}
\end{frame}