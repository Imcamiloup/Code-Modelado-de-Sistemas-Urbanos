\begin{frame}
  \frametitle{Tratamiento del Modelo}


  \begin{block}{Decisión Metodológica: Tratamiento de Variables Exógenas}
    \begin{itemize}
      \item Se adoptó el tratamiento de las variables $U(t)$ y $E(t)$ como \textbf{fuerzas externas} con evolución temporal lenta.
      \item Las ecuaciones $ \frac{dU}{dt} = 0 $ y $ \frac{dE}{dt} = 0 $ permiten que el sistema tenga infinitos puntos de equilibrio.
    \end{itemize}
  \end{block}

  \vspace{0.5cm}

  \begin{block}{Fundamentación Física}
    \begin{itemize}
      \item La huella urbana $U(t)$ cambia lentamente, representando la infraestructura física.
      \item La infraestructura ecológica $E(t)$ evoluciona a lo largo de décadas, como parques y zonas protegidas.
      \item Se contrastan con las dinámicas sociales rápidas como el crecimiento poblacional y las decisiones políticas.
    \end{itemize}
  \end{block}

\end{frame}