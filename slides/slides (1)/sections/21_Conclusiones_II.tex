% Diapositiva 6: Conclusiones II - Implicaciones de Política
\begin{frame}{Conclusiones II: Implicaciones de Política Urbana}
\small
\textbf{4. Bienestar final subóptimo}
\begin{itemize}
  \item $W_{\text{eq}} = -0.09$ (z-score) $\Rightarrow$ por debajo de media histórica
  \item Causas: $\delta_W = 1.0$ (desigualdad penaliza fuerte) + $\gamma_W = -1.0$ (amortiguamiento sin crecimiento endógeno)
  \item \textcolor{blue}{\textbf{Estrategia:}} Reducir $\delta_W$ (políticas redistributivas) o aumentar $\beta_W$ (inversión en infraestructura U, E)
\end{itemize}

\textbf{5. Palancas de intervención identificadas}
\begin{itemize}
  \item \textbf{Aumentar $K$:} Expandir capacidad (infraestructura) $\Rightarrow$ permite $P$ mayor sin colapso
  \item \textbf{Fortalecer $\beta_F$:} Vincular bienestar a reducción de desigualdad (programas sociales focalizados)
  \item \textbf{Moderar $\delta_W$:} Diseñar política que amortigüe impacto negativo de $F$ en $W$ (redes de protección)
\end{itemize}

\textbf{6. Validación del modelo}
\begin{itemize}
  \item ✅ Físicamente plausible: atractor único, convergencia estable
  \item ✅ Coherente con datos: $R^2_P = 0.96$ (población bien ajustada)
  \item ⚠️ Mejora pendiente: $R^2_F = 0.17$, $R^2_W = 0.27$ (alta volatilidad residual) $\Rightarrow$ incorporar no linealidades ($F^2$, retardos temporales)
\end{itemize}
\end{frame}