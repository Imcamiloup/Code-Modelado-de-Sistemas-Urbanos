% Diapositiva 4: Interpretación Diagramas de Fase
\begin{frame}{Interpretación de Diagramas de Fase}
\small
\textbf{Plano P-W (F=0):}
\begin{itemize}
  \item Flujo radial hacia PF2 (rojo) desde todas las direcciones 
  \item PF1 (amarillo) actúa como punto de paso transitorio (manifold inestable)
  \item Acoplamiento fuerte: $\alpha_W P$ en ecuación de $W$ induce correlación positiva (más población $\rightarrow$ más bienestar, hasta saturación)
\end{itemize}

\textbf{Plano P-F (W=0):}
\begin{itemize}
  \item Líneas circulares sugieren amortiguamiento en ciclo P-F (sin ciclo límite sostenido)
  \item $\gamma_F = -1.69$ (negativo) $\Rightarrow$ desigualdad alta reduce población (presión social, migración)
  \item Convergencia hacia P alto, F bajo (reducción estructural de desigualdad)
\end{itemize}

\textbf{Plano F-W (P fijo):}
\begin{itemize}
  \item Campo espiral hacia equilibrio (acoplamiento bidireccional débil: $\beta_F=0.01$ pequeño)
  \item $\delta_W=1.0$ (fuerte) $\Rightarrow$ desigualdad penaliza severamente bienestar
\end{itemize}
\end{frame}