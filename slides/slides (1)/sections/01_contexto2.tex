\begin{frame}
  \frametitle{Contexto y Motivación}

  

  \begin{itemize}
    \item Los modelos matemáticos permiten simular escenarios de sostenibilidad y de colapso ecológico.
    \item Son fundamentales en la planificación de políticas públicas efectivas para la gestión de los recursos urbanos, la protección del medio ambiente y el bienestar social.
    \item El trabajo en este proyecto se basa en esta premisa, integrando la población, la huella urbana y la estructura ecológica mediante un sistema de Ecuaciones Diferenciales Ordinarias (EDOs) adaptado para Bogotá.
  \end{itemize}
  
  \begin{block}{Objetivos}
    \begin{itemize}
      \item Desarrollar un modelo matemático que integre la dinámica de población, huella urbana, bienestar social y sostenibilidad ecológica.
      \item Utilizar este modelo para evaluar políticas públicas orientadas hacia un desarrollo urbano sostenible.
    \end{itemize}
  \end{block}
\end{frame}